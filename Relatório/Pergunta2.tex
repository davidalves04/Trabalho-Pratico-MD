\chapter{Pergunta 2}

Primeiro definimos o valor de $\beta$ como o último algarismo do número de estudante. Considerando que o número de estudante seja 8240266, temos que:

\[
\beta = 6
\]

Agora, escolhemos $n = 30$, de forma que satisfaça a condição $50 + \beta < 2n < 100 - \beta$. Ou seja, temos:

\[
50 + 6 < 2n < 100 - 6 \quad \Rightarrow \quad 56 < 2n < 94 \quad \Rightarrow \quad 28 < n < 47
\]



\begin{enumerate}


    \item[a)]
    Queremos calcular o somatório:
    \[
    \sum_{j=\beta+2}^{n} \left( \frac{-2\beta-1}{5} \right)^j
    \]
    Para \(\beta = 6\) e \(n = 30\), temos:
    \[
    \sum_{j= \beta+2}^{n} \left( \frac{-2\beta-1}{5} \right)^j = \sum_{j= 6+2}^{30} \left( \frac{-2 \cdot 6 - 1}{5} \right)^j
    \]
    Ou seja:
    \[
    \sum_{j= 8}^{30} \left( \frac{-12 - 1}{5} \right)^j
    \]


\section*{Resultado utilizando  código scilab:}
     \[
    \sum_{j= 8}^{30} \left( \frac{-12 - 1}{5} \right)^j = 2.032 \times 10^{12}
    \]


   \item[b)]
   Queremos calcular o somatório:
   \[
\prod_{i \in C} \left( \frac{\beta + 1}{i - 1} \right)^4, \quad C = \left\{ 5m \in \mathbb{Z}: m = 1, \dots, M \right\}, \quad M = \min \left( 5 + \beta, \left\lceil \frac{100}{\beta + 1} \right\rceil \right)
\]
    Para \(\beta = 6\) e \(n = 30\), temos:
     \[
\prod_{i \in C} \left( \frac{6 + 1}{i - 1} \right)^4, \quad C = \left\{ 5m \in \mathbb{Z}: m = 1, \dots, M \right\}, \quad M = \min \left( 5 + \beta, \left\lceil \frac{100}{\beta + 1} \right\rceil \right) 
\]
 = \[
\prod_{i \in C} \left( \frac{7}{i - 1} \right)^4, \quad C = \left\{ 5m \in \mathbb{Z}: m = 1, \dots, M \right\}, \quad M = \min \left( 5 + \beta, \left\lceil \frac{100}{\beta + 1} \right\rceil \right) 
\]

\section*{Resultado utilizando  código scilab:}

\[
\prod_{i \in C} \left( \frac{7}{i - 1} \right)^4, \quad C = \left\{ 5m \in \mathbb{Z}: m = 1, \dots, M \right\}, \quad M = \min \left( 5 + \beta, \left\lceil \frac{100}{\beta + 1} \right\rceil \right) = 1338.5940
\]
 



      \item[c)]

    Queremos calcular o somatório:

      \[
\prod_{k=1}^{n-15} \left( 3 \times \sum_{j=n-5}^{n} \left( \left\lfloor 1 + \frac{j + k}{200} \right\rfloor - \left\lceil \frac{6!}{\beta + 1} \right\rceil \right) \right)
\]

    Para \(\beta = 6\) e \(n = 30\), temos:
  \[
\prod_{k=1}^{30-15} \left( 3 \times \sum_{j=30-5}^{30} \left( \left\lfloor 1 + \frac{j + k}{200} \right\rfloor - \left\lceil \frac{6!}{6 + 1} \right\rceil \right) \right)
\]
 = \[
\prod_{k=1}^{15} \left( 3 \times \sum_{j=25}^{30} \left( \left\lfloor 1 + \frac{j + k}{200} \right\rfloor - \left\lceil \frac{6!}{7} \right\rceil \right) \right)
\]

\section*{Resultado utilizando  código scilab:}
\[
\prod_{k=1}^{15} \left( 3 \times \sum_{j=25}^{30} \left( \left\lfloor 1 + \frac{j + k}{200} \right\rfloor - \left\lceil \frac{6!}{7} \right\rceil \right) \right) =  1.006 \times 10^{38}
\] 



    
\end{enumerate}


\section*{Código Scilab Utilizado:}

\subsection*{Parâmetros Iniciais}
\begin{verbatim}
beta = 6;  // Último algarismo do número do estudante
n = 30;    // Escolha de n tal que 50 + beta < 2n < 100 - beta
\end{verbatim}

\subsection*{Alínea (a)}
\begin{verbatim}
a = 0;
for j = beta + 2 : n
    a = a + ((-2*beta - 1)/5)^j;
end
disp("Resultado da alínea (a):", a);
\end{verbatim}

\subsection*{Alínea (b)}
\begin{verbatim}
M = min(5 + beta, floor(100 / (beta + 1)));
b = 0;
for i = 1:M
    b = b + ((beta + 1)/i - 1)^4;
end
disp("Resultado da alínea (b):", b);
\end{verbatim}

\subsection*{Alínea (c)}
\begin{verbatim}
c = 1;
const = floor(factorial(6)/(beta + 1));  // valor constante
for k = 1:(n - 15)
    soma = 0;
    for j = n - 5 : n
        soma = soma + (1 + ((j + k)/200)*const);
    end
    c = c * (3 * soma);
end
disp("Resultado da alínea (c):", c);
\end{verbatim}