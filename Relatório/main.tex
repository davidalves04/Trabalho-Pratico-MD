\documentclass[10pt,a4paper,titlepage,twoside,openright]{report}
%packages
\usepackage{makeidx}
\usepackage[portuges]{babel}
\usepackage{fancyhdr}
\usepackage{lipsum} % for dummy text
\usepackage{enumitem}
\setlist{nosep} % or \setlist{noitemsep} to leave space around whole list
% Figures
\usepackage[usenames,dvipsnames]{color}
\usepackage{graphicx}
\usepackage{caption}
\usepackage{multirow}
\usepackage{framed}
% Required for custom colors
\usepackage[usenames,dvipsnames]{xcolor}
% hyperlinks
\usepackage{url}
\usepackage{hyperref}
% Math
\usepackage{amssymb,amsbsy,amsmath}
\usepackage{amsthm} %para poder definir newtheorem
\usepackage{amsfonts} %para poder fazer o R, Z etc \mathbb{R}	\mathbb{Z} 

%estilo de página
\pagestyle{fancy}
			\lhead{Trabalho Prático de Matemática Discreta}
			\chead{ }
			\rhead{	\includegraphics[height=0.45cm]{Figuras/PPorto-peq}}
			\lfoot{\hrule }
			\cfoot{}
			\rfoot{\hrule \vspace*{5pt} \thepage}
	
%\renewcommand{\chaptermark}[1]{olaallalal}


\linespread{1.5}    							% espaçamento entre linhas
\setlength{\parindent}{0pt} 					% espaço horizontal aquando a mudança de paragrafo
\setlength{\parskip}{1ex plus 0.5ex minus 0.2ex}% espaço vertical entre parágrafos
\setlength{\headheight}{17pt}
\newcommand{\Lower}[1]{\smash {\lower 1.5ex \hbox{#1}}}

\usepackage{titlesec}
\titleformat
{\chapter} % command
[display] % shape
{\bfseries\Huge%\itshape
} % format
{%Story No. \ \thechapter
} % label
{0.5ex} % sep
{
%    \rule{\textwidth}{1pt}
    \vspace{1ex}
    %\centering
} % before-code
[
%\vspace{-0.5ex}%
%\rule{\textwidth}{0.3pt}
] % after-code
%__________________________________________________________________________________________________________
\begin{document}



\begin{titlepage}%

	\begin{flushright}
	\includegraphics[height=1.0cm]{Figuras/logo_ESTG}
	\end{flushright}
	
	\vfill
	
  \begin{center}
  \Large{Trabalho Prático de }\\
  \Huge{\textbf{Matemática Discreta}}\\
  \Huge{\textbf{2024/2025}}
  	\end{center}  
  	
  	\vfill  	
  	\vfill
  	\vfill  	
  	\vfill
  	
  	Trabalho Elaborado por:\\
  	\textbf{Grupo M}
  	\begin{itemize}
  	    \item[] 
  	    8240231 David Sérgio Ferreira Alves
  	    \item []
  	    8240216 Pedro Afonso Farinha Gomes Lima Paraty
  	    \item[]
  	    8240266 Gabriel Alexandre Meireles Moreira
  	    \item[]
  	    8240839 Samuel Silva da Cunha
  	\end{itemize}
  	Curso de:\\
 \textsf{Licenciatura em Segurança Informática e Redes de Computadores}
	
  	\vfill  	
  	\vfill
  	
  	
  	
  	
  	Docentes:\\
  Eliana Costa e Silva (eos@estg.ipp.pt) \\
  Isabel Cristina Duarte (icd@estg.ipp.pt) 

	\begin{center}
  Felgueiras, \today
  	\end{center}  
        
\end{titlepage}%
\setcounter{footnote}{0}

%Página em branco - verso da Capa
\newpage\thispagestyle{empty}

%Índice
\tableofcontents
% Lista de Figuras
\listoffigures
% Lista de Tabelas
\listoftables

%Página em branco - verso do índice
\newpage\thispagestyle{empty}

%__________________________________________________________________________________________________________
% Pergunta 1
\chapter{Pergunta 1}

\subsection*{Conjuntos Escolhidos}

Seja o conjunto universo:
\[
U = \{1, 2, 3, \ldots, 20\}
\]

Escolheram-se os seguintes subconjuntos:
\[
A = \{3, 5, 9, 11, 13, 17, 19\} \quad (\#A = 7)
\]
\[
B = \{2, 3, 4, 5, 6, 7, 8, 9, 10, 11, 12, 13, 14, 15, 16, 17, 18\} \quad (\#B = 17)
\]

Estes conjuntos cumprem:
\begin{itemize}
    \item \(5 \leq \#A < 10\)
    \item \(\#B > 15\)
    \item \(A \neq B\), \(A \neq U\), \(B \neq U\)
\end{itemize}

\subsection*{Resolução das Alíneas}

\begin{enumerate}
    \item[a)] Cardinalidade:
    \[
    \#A = 7, \quad \#B = 17
    \]

    \item[b)] Complemento de \(A\) em relação a \(U\):
    \[
    \bar{A} = U \setminus A = \{1, 2, 4, 6, 7, 8, 10, 12, 14, 15, 16, 18, 20\}
    \]

    \item[c)] União:
    \[
    A \cup B = \{2, 3, 4, 5, 6, 7, 8, 9, 10, 11, 12, 13, 14, 15, 16, 17, 18, 19\}
    \]

    \item[d)] Interseção:
    \[
    A \cap B = \{3, 5, 9, 11, 13, 17\}
    \]

    \item[e)] Diferença \(B - A\):
    \[
    B - A = \{2, 4, 6, 7, 8, 10, 12, 14, 15, 16, 18\}
    \]

    \item[f)] Diferença simétrica \(A \oplus B\):
    \[
    (A \setminus B) \cup (B \setminus A) = \{2, 4, 6, 7, 8, 10, 12, 14, 15, 16, 18, 19\}
    \]

    \item[g)] \(A \oplus \bar{B} \cup (A - B)\):

    Calculando primeiro \(\bar{B} = U \setminus B = \{1, 19, 20\}\)

    \[
    A \oplus \bar{B} = \{1, 3, 5, 9, 11, 13, 17, 20\}
    \]

    \[
    A - B = \{19\}
    \]

    \[
    Resultado final: \{1, 3, 5, 9, 11, 13, 17, 19, 20\}
    \]

    \item[h)] Produto cartesiano \(B \times A\):

    Apenas os 5 primeiros pares:
    \[
    \{(2,3), (2,5), (2,9), (2,11), (2,13), \ldots\}
    \]

    \item[i)] Produto cartesiano \(A \times A \times A\):

    Apenas os 5 primeiros trios:
    \[
    \{(3,3,3), (3,3,5), (3,3,9), (3,3,11), (3,3,13), \ldots\}
    \]
\end{enumerate}

\subsection*{Código Scilab Usado}

\begin{lstlisting}[language=Scilab, basicstyle=\small\ttfamily, breaklines=true]
U = 1:20;
A = [3, 5, 9, 11, 13, 17, 19];
B = [2:18];

length(A)
length(B)

Ac = setdiff(U, A)
unionAB = union(A, B)
interAB = intersect(A, B)
diffBA = setdiff(B, A)

diffAB = setdiff(A, B)
diffBA = setdiff(B, A)
symDiff = union(diffAB, diffBA)

Bc = setdiff(U, B)
symDiff2 = union(setdiff(A, Bc), setdiff(Bc, A))
resG = union(symDiff2, diffAB)

cartProd = [];
for i = 1:length(B)
    for j = 1:length(A)
        cartProd($+1,:) = [B(i), A(j)];
    end
end

A3 = [];
for i = 1:length(A)
    for j = 1:length(A)
        for k = 1:length(A)
            A3($+1,:) = [A(i), A(j), A(k)];
        end
    end
end
\end{lstlisting}

%__________________________________________________________________________________________________________
% Pergunta 2
\chapter{Pergunta 2}

Primeiro definimos o valor de $\beta$ como o último algarismo do número de estudante. Considerando que o número de estudante seja 8240266, temos que:

\[
\beta = 6
\]

Agora, escolhemos $n = 30$, de forma que satisfaça a condição $50 + \beta < 2n < 100 - \beta$. Ou seja, temos:

\[
50 + 6 < 2n < 100 - 6 \quad \Rightarrow \quad 56 < 2n < 94 \quad \Rightarrow \quad 28 < n < 47
\]



\begin{enumerate}


    \item[a)]
    Queremos calcular o somatório:
    \[
    \sum_{j=\beta+2}^{n} \left( \frac{-2\beta-1}{5} \right)^j
    \]
    Para \(\beta = 6\) e \(n = 30\), temos:
    \[
    \sum_{j= \beta+2}^{n} \left( \frac{-2\beta-1}{5} \right)^j = \sum_{j= 6+2}^{30} \left( \frac{-2 \cdot 6 - 1}{5} \right)^j
    \]
    Ou seja:
    \[
    \sum_{j= 8}^{30} \left( \frac{-12 - 1}{5} \right)^j
    \]


\section*{Resultado utilizando  código scilab:}
     \[
    \sum_{j= 8}^{30} \left( \frac{-12 - 1}{5} \right)^j = 2.032 \times 10^{12}
    \]


   \item[b)]
   Queremos calcular o somatório:
   \[
\prod_{i \in C} \left( \frac{\beta + 1}{i - 1} \right)^4, \quad C = \left\{ 5m \in \mathbb{Z}: m = 1, \dots, M \right\}, \quad M = \min \left( 5 + \beta, \left\lceil \frac{100}{\beta + 1} \right\rceil \right)
\]
    Para \(\beta = 6\) e \(n = 30\), temos:
     \[
\prod_{i \in C} \left( \frac{6 + 1}{i - 1} \right)^4, \quad C = \left\{ 5m \in \mathbb{Z}: m = 1, \dots, M \right\}, \quad M = \min \left( 5 + \beta, \left\lceil \frac{100}{\beta + 1} \right\rceil \right) 
\]
 = \[
\prod_{i \in C} \left( \frac{7}{i - 1} \right)^4, \quad C = \left\{ 5m \in \mathbb{Z}: m = 1, \dots, M \right\}, \quad M = \min \left( 5 + \beta, \left\lceil \frac{100}{\beta + 1} \right\rceil \right) 
\]

\section*{Resultado utilizando  código scilab:}

\[
\prod_{i \in C} \left( \frac{7}{i - 1} \right)^4, \quad C = \left\{ 5m \in \mathbb{Z}: m = 1, \dots, M \right\}, \quad M = \min \left( 5 + \beta, \left\lceil \frac{100}{\beta + 1} \right\rceil \right) = 1338.5940
\]
 



      \item[c)]

    Queremos calcular o somatório:

      \[
\prod_{k=1}^{n-15} \left( 3 \times \sum_{j=n-5}^{n} \left( \left\lfloor 1 + \frac{j + k}{200} \right\rfloor - \left\lceil \frac{6!}{\beta + 1} \right\rceil \right) \right)
\]

    Para \(\beta = 6\) e \(n = 30\), temos:
  \[
\prod_{k=1}^{30-15} \left( 3 \times \sum_{j=30-5}^{30} \left( \left\lfloor 1 + \frac{j + k}{200} \right\rfloor - \left\lceil \frac{6!}{6 + 1} \right\rceil \right) \right)
\]
 = \[
\prod_{k=1}^{15} \left( 3 \times \sum_{j=25}^{30} \left( \left\lfloor 1 + \frac{j + k}{200} \right\rfloor - \left\lceil \frac{6!}{7} \right\rceil \right) \right)
\]

\section*{Resultado utilizando  código scilab:}
\[
\prod_{k=1}^{15} \left( 3 \times \sum_{j=25}^{30} \left( \left\lfloor 1 + \frac{j + k}{200} \right\rfloor - \left\lceil \frac{6!}{7} \right\rceil \right) \right) =  1.006 \times 10^{38}
\] 



    
\end{enumerate}


\section*{Código Scilab Utilizado:}

\subsection*{Parâmetros Iniciais}
\begin{verbatim}
beta = 6;  // Último algarismo do número do estudante
n = 30;    // Escolha de n tal que 50 + beta < 2n < 100 - beta
\end{verbatim}

\subsection*{Alínea (a)}
\begin{verbatim}
a = 0;
for j = beta + 2 : n
    a = a + ((-2*beta - 1)/5)^j;
end
disp("Resultado da alínea (a):", a);
\end{verbatim}

\subsection*{Alínea (b)}
\begin{verbatim}
M = min(5 + beta, floor(100 / (beta + 1)));
b = 0;
for i = 1:M
    b = b + ((beta + 1)/i - 1)^4;
end
disp("Resultado da alínea (b):", b);
\end{verbatim}

\subsection*{Alínea (c)}
\begin{verbatim}
c = 1;
const = floor(factorial(6)/(beta + 1));  // valor constante
for k = 1:(n - 15)
    soma = 0;
    for j = n - 5 : n
        soma = soma + (1 + ((j + k)/200)*const);
    end
    c = c * (3 * soma);
end
disp("Resultado da alínea (c):", c);
\end{verbatim}

%__________________________________________________________________________________________________________
% Pergunta 3
\chapter{Pergunta 3}
\section*{Alínea A}

Foram geradas 20 coordenadas tridimensionais (X, Y, Z) aleatórias no intervalo \([-27, 27]\), utilizando a função \texttt{rand()}.

As coordenadas obtidas foram:

\begin{verbatim}
   13.421736  -4.8380814   5.8564422
   23.016656   3.6029409   3.8428503
  -20.253562  12.3078000 -12.5400610
  -13.218089   6.7601487 -20.7499460
    6.031822   9.6333639  -9.0714855
  -25.602966   0.9421284  -5.8488845
  -13.966893   0.3479484  -4.1250492
  -11.373871 -22.2051660   6.5495615
   -8.343084  11.1502850   1.1419515
  -11.499835   8.1150934 -22.2407920
   -2.706678  12.0271670  21.4747000
    0.370086   1.2580726   3.2235170
    3.333458  -1.7184959  15.0905520
   -4.012889 -13.7075730  22.8394750
  -21.595973  -1.7376222  -5.6673126
    4.293108 -12.8080030  -3.4506722
  -13.011687  -4.6489286  -7.5603900
   10.329055  14.3470370  -7.7076919
   14.544359   2.5792231 -21.8036410
    8.431460 -13.7940900   1.5288673
\end{verbatim}

\section*{Alínea B}

A matriz de adjacência resultante (valores arredondados para 7 casas decimais) contém a maioria dos elementos a zero, indicando poucas conexões no grafo. Exemplo:

\begin{verbatim}
... 
(10,4) = 2.6478092
(15,6) = 4.8239020
(4,10) = 2.6478092
(6,15) = 4.8239020
...
\end{verbatim}
\medskip

\noindent
A matriz apresenta valores reais não nulos, que correspondem às distâncias entre pares de pontos tridimensionais conectados por arestas. Tratando-se assim de uma \textbf{matriz de adjacência ponderada}, ou seja, uma matriz de pesos, onde o valor de cada entrada representa o custo (neste caso, a distância euclidiana) da ligação entre dois nós. As entradas a zero indicam ausência de conexão entre os respetivos pares de vértices.

Todos os outros elementos da matriz são zero, o que indica a existência de apenas duas arestas no grafo (ambas bidirecionais).

\section*{Alínea C}
A determinação do caminho mais curto entre dois nós do grafo foi realizada com o algoritmo de Dijkstra. Com base nos números de estudante \textbf{8240231} e \textbf{8240266}, foram utilizados os valores de \(\beta = 1\) e \(\sigma = 66\), que foram determinados pelo último algarismo de cada número.
No entanto, como , foi aplicado o seguinte critério de ajuste:
\newline
\(\sigma = 10\) (caso \(\sigma > 20\)).

Como o valor da distância entre os nós foi infinito, conclui-se que não existe caminho entre os dois pontos selecionados:
\begin{center}
\textit{"Não existe caminho entre os pontos selecionados."}
\end{center}

\bigskip

\noindent Assim, o grafo gerado com os dados definidos não contém um caminho entre os vértices indicados.


\end{document}