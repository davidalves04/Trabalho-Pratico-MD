\chapter{Pergunta 1}

\subsection*{Conjuntos Escolhidos}

Seja o conjunto universo:
\[
U = \{1, 2, 3, \ldots, 20\}
\]

Escolheram-se os seguintes subconjuntos:
\[
A = \{3, 5, 9, 11, 13, 17, 19\}
\]
\[
B = \{2, 3, 4, 5, 6, 7, 8, 9, 10, 11, 12, 13, 14, 15, 16, 17, 18\}
\]

Estes conjuntos cumprem:
\begin{itemize}
    \item \(5 \leq \#A < 10\)
    \item \(\#B > 15\)
    \item \(A \neq B\), \(A \neq U\), \(B \neq U\)
\end{itemize}

\newpage
\subsection*{Resolução das Alíneas}

\begin{enumerate}
    \item[a)] Cardinalidade (número de elementos do conjunto):
    \[
    \#A = 7, \quad \#B = 17
    \]

    \item[b)] Complemento de B (conjunto dos elementos de \(U\) que não pertencem a \(B\)):
    \[
    \bar{B} = \{1, 19, 20\}
    \]
    \item[c)] União (conjunto dos elementos que pertencem a \(A\), a \(B\) (ou a ambos)):
    \[
    A \cup B = \{2, 3, 4, 5, 6, 7, 8, 9, 10, 11, 12, 13, 14, 15, 16, 17, 18, 19\}
    \]

    \item[d)] Interseção (conjunto dos elementos que pertencem simultaneamente a \(A\) e a \(B\)):
    \[
    A \cap B = \{3, 5, 9, 11, 13, 17\}
    \]

    \item[e)] Diferença \(B - A\) (elementos que estão em \(B\), mas não estão em \(A\)):
    \[
    B - A = \{2, 4, 6, 7, 8, 10, 12, 14, 15, 16, 18\}
    \]

    \item[f)] Diferença simétrica \(A \oplus B\) (elementos que pertencem a um só dos conjuntos, mas não a ambos):
    \[
    (A \setminus B) \cup (B \setminus A) = \{2, 4, 6, 7, 8, 10, 12, 14, 15, 16, 18, 19\}
    \]

    \item[g)] \(\overline{A \oplus {B}} \cup (A - B)\) (complemento da diferença simétrica, unido com a diferença \(A - B\)):

    Calculou-se primeiro o  $\overline{A \oplus B}$:

    \[
   \overline{A \oplus {B}} =U -  A \oplus {B} =\{1,3,5,9,11,13,17,20\}
    \]
De seguida:
    \[
    A - B = \{19\}
    \]
Por fim, realizou-se a reunião

    \[
    Resultado final: \{1,3,5,9,11,13,17,19,20\}
    \]

    \item[h)] Produto cartesiano \(B \times A\) (conjunto de pares ordenados onde o primeiro elemento pertence a \(B\) e o segundo a \(A\)):
    
    Como existem no total 119 pares (17x7), selecionaram-se apenas os 5 primeiros pares:
    \[
    \{(2,3), (2,5), (2,9), (2,11), (2,13), \ldots\}
    \]

    \item[i)] Produto cartesiano \(A \times A \times A\) (conjunto de trios ordenados formados com elementos de \(A\)):

      Como existem no total 343 trios (7$^{3}$), selecionaram-se apenas os 5 primeiros trios:
    \[
    \{(3,3,3), (3,3,5), (3,3,9), (3,3,11), (3,3,13), \ldots\}
    \]
\end{enumerate}

