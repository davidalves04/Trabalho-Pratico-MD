\chapter{Pergunta 3}
\section*{Alínea A}

Foram geradas 20 coordenadas tridimensionais (X, Y, Z) aleatórias no intervalo \([-27, 27]\), utilizando a função \texttt{rand()}.

As coordenadas obtidas foram:

\begin{verbatim}
   13.421736  -4.8380814   5.8564422
   23.016656   3.6029409   3.8428503
  -20.253562  12.3078000 -12.5400610
  -13.218089   6.7601487 -20.7499460
    6.031822   9.6333639  -9.0714855
  -25.602966   0.9421284  -5.8488845
  -13.966893   0.3479484  -4.1250492
  -11.373871 -22.2051660   6.5495615
   -8.343084  11.1502850   1.1419515
  -11.499835   8.1150934 -22.2407920
   -2.706678  12.0271670  21.4747000
    0.370086   1.2580726   3.2235170
    3.333458  -1.7184959  15.0905520
   -4.012889 -13.7075730  22.8394750
  -21.595973  -1.7376222  -5.6673126
    4.293108 -12.8080030  -3.4506722
  -13.011687  -4.6489286  -7.5603900
   10.329055  14.3470370  -7.7076919
   14.544359   2.5792231 -21.8036410
    8.431460 -13.7940900   1.5288673
\end{verbatim}

\section*{Alínea B}

A matriz de adjacência resultante (valores arredondados para 7 casas decimais) contém a maioria dos elementos a zero, indicando poucas conexões no grafo. Exemplo:

\begin{verbatim}
... 
(10,4) = 2.6478092
(15,6) = 4.8239020
(4,10) = 2.6478092
(6,15) = 4.8239020
...
\end{verbatim}

Todos os outros elementos da matriz são zero, o que indica a existência de apenas duas arestas no grafo (ambas bidirecionais).

\section*{Alínea C}
A determinação do caminho mais curto entre dois nós do grafo foi realizada com o algoritmo de Dijkstra. Com base nos números de estudante \textbf{8240231} e \textbf{8240266}, foram utilizados os valores de \(\beta = 1\) e \(\sigma = 66\), que foram determinados pelo último algarismo de cada número.
No entanto, como , foi aplicado o seguinte critério de ajuste:
\newline
\(\sigma = 10\) (caso \(\sigma > 20\)).

Como o valor da distância entre os nós foi infinito, conclui-se que não existe caminho entre os dois pontos selecionados:
\begin{center}
\textit{"Não existe caminho entre os pontos selecionados."}
\end{center}

\bigskip

\noindent Assim, o grafo gerado com os dados definidos não contém um caminho entre os vértices indicados.
