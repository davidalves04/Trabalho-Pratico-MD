\chapter{Pergunta 2}

Primeiro definimos o valor de $\beta$ como o último algarismo do número de estudante. Considerando que o número de estudante seja 8240266, temos que:

\[
\beta = 6
\]

Agora, escolhemos $n = 30$, de forma que satisfaça a condição $50 + \beta < 2n < 100 - \beta$. Ou seja, temos:

\[
50 + 6 < 2n < 100 - 6 \quad \Rightarrow \quad 56 < 2n < 94 \quad \Rightarrow \quad 28 < n < 47
\]



\begin{enumerate}


    \item[a)]
    Queremos calcular o somatório:
    \[
    \sum_{j=\beta+2}^{n} \left( \frac{-2\beta-1}{5} \right)^j
    \]
    Para \(\beta = 6\) e \(n = 30\), temos:
    \[
    \sum_{j= \beta+2}^{n} \left( \frac{-2\beta-1}{5} \right)^j = \sum_{j= 6+2}^{30} \left( \frac{-2 \cdot 6 - 1}{5} \right)^j
    \]
    Ou seja:
    \[
    \sum_{j= 8}^{30} \left( \frac{-13}{5} \right)^j
    \]


\section*{Resultado utilizando  código scilab:}
     \[
    \sum_{j= 8}^{30} \left( \frac{-13}{5} \right)^j = 2.032 \times 10^{12}
    \]
A sequência dos termos forma uma progressão geométrica com razão $\frac{-13}{5}$


   \item[b)]
   Queremos calcular o somatório:
   \[
\prod_{i \in C} \left( \frac{\beta + 1}{i } -1\right)^4, \quad C = \left\{ 5m \in \mathbb{Z}: m = 1, \dots, M \right\}, \quad M = \min \left( 5 + \beta, \left\lceil \frac{100}{\beta + 1} \right\rceil \right)
\]
    Para \(\beta = 6\) e \(n = 30\), temos:
     \[
\prod_{i \in C} \left( \frac{6 + 1}{i } -1\right)^4, \quad C = \left\{ 5m \in \mathbb{Z}: m = 1, \dots, M \right\}, \quad M = \min \left( 5 + \beta, \left\lceil \frac{100}{\beta + 1} \right\rceil \right) 
\]
 = \[
\prod_{i \in C} \left( \frac{7}{i }-1 \right)^4, \quad C = \left\{ 5m \in \mathbb{Z}: m = 1, \dots, M \right\}, \quad M = \min \left( 5 + \beta, \left\lceil \frac{100}{\beta + 1} \right\rceil \right) 
\]

\section*{Resultado utilizando  código scilab:}

\[
\prod_{i \in C} \left( \frac{7}{i } -1 \right)^4, \quad C = \left\{ 5m \in \mathbb{Z}: m = 1, \dots, M \right\}, \quad M = \min \left( 5 + \beta, \left\lceil \frac{100}{\beta + 1} \right\rceil \right) = 8.513^{-9}
\]

 \vspace{2em}
      \item[c)]

    Queremos calcular o somatório:

      \[
\prod_{k=1}^{n-15} \left( 3 \times \sum_{j=n-5}^{n} \left( \left\lfloor 1 + \frac{j + k}{200} \right\rfloor - \left\lceil \frac{6!}{\beta + 1} \right\rceil \right) \right)
\]

    Para \(\beta = 6\) e \(n = 30\), temos:
  \[
\prod_{k=1}^{30-15} \left( 3 \times \sum_{j=30-5}^{30} \left( \left\lfloor 1 + \frac{j + k}{200} \right\rfloor - \left\lceil \frac{6!}{6 + 1} \right\rceil \right) \right)
\]
 = \[
\prod_{k=1}^{15} \left( 3 \times \sum_{j=25}^{30} \left( \left\lfloor 1 + \frac{j + k}{200} \right\rfloor - \left\lceil \frac{6!}{7} \right\rceil \right) \right)
\]

\section*{Resultado utilizando  código scilab:}
\[
\prod_{k=1}^{15} \left( 3 \times \sum_{j=25}^{30} \left( \left\lfloor 1 + \frac{j + k}{200} \right\rfloor - \left\lceil \frac{6!}{7} \right\rceil \right) \right) = -9.080 \times 10^{48}
\] 

\end{enumerate}


